% !TEX TS-program = pdflatex
% !TEX encoding = UTF-8 Unicode

% This is a simple template for a LaTeX document using the "article" class.
% See "book", "report", "letter" for other types of document.

\documentclass[11pt]{article} % use larger type; default would be 10pt

\usepackage[utf8]{inputenc} % set input encoding (not needed with XeLaTeX)

%%% Examples of Article customizations
% These packages are optional, depending whether you want the features they provide.
% See the LaTeX Companion or other references for full information.

%%% PAGE DIMENSIONS
\usepackage{geometry} % to change the page dimensions
\geometry{a4paper} % or letterpaper (US) or a5paper or....
% \geometry{margin=2in} % for example, change the margins to 2 inches all round
% \geometry{landscape} % set up the page for landscape
%   read geometry.pdf for detailed page layout information

\usepackage{graphicx} % support the \includegraphics command and options

% \usepackage[parfill]{parskip} % Activate to begin paragraphs with an empty line rather than an indent

%%% PACKAGES
\usepackage{booktabs} % for much better looking tables
\usepackage{array} % for better arrays (eg matrices) in maths
\usepackage{paralist} % very flexible & customisable lists (eg. enumerate/itemize, etc.)
\usepackage{verbatim} % adds environment for commenting out blocks of text & for better verbatim
\usepackage{subfig} % make it possible to include more than one captioned figure/table in a single float
% These packages are all incorporated in the memoir class to one degree or another...

%%% HEADERS & FOOTERS
\usepackage{fancyhdr} % This should be set AFTER setting up the page geometry
\pagestyle{fancy} % options: empty , plain , fancy
\renewcommand{\headrulewidth}{0pt} % customise the layout...
\lhead{}\chead{}\rhead{}
\lfoot{}\cfoot{\thepage}\rfoot{}

%%% SECTION TITLE APPEARANCE
\usepackage{sectsty}
\allsectionsfont{\sffamily\mdseries\upshape} % (See the fntguide.pdf for font help)
% (This matches ConTeXt defaults)

%%% ToC (table of contents) APPEARANCE
\usepackage[nottoc,notlof,notlot]{tocbibind} % Put the bibliography in the ToC
\usepackage[titles,subfigure]{tocloft} % Alter the style of the Table of Contents
\renewcommand{\cftsecfont}{\rmfamily\mdseries\upshape}
\renewcommand{\cftsecpagefont}{\rmfamily\mdseries\upshape} % No bold!

\usepackage{listings}

%%% END Article customizations

%%% The "real" document content comes below...

\title{GMQL}
\author{Simone Mosciatti}
%\date{} % Activate to display a given date or no date (if empty),
         % otherwise the current date is printed 

\begin{document}
\maketitle

\section{Exercise 1}

\begin{lstlisting}
# Among the different epigenetic signals, H3K27ac and H3K27me3 are generally 
# associated with active and repressed chromatin regions, respectively. 
# Considering H3K4me1 in cell line A549 and the aforementioned signals 
# (H3K27ac and H3K27me3) in broadPeak format, under ethanol treatment (EtOH)

K4me1EtOH  = SELECT(cell == 'A549' AND treatment == 'EtOH_0.02pct' AND antibody == 'H3K4me1' ) HG19_ENCODE_BROAD;
K27acEtOH  = SELECT(cell == 'A549' AND treatment == 'EtOH_0.02pct' AND antibody == 'H3K27ac' ) HG19_ENCODE_BROAD;
K27me3EtOH = SELECT(cell == 'A549' AND treatment == 'EtOH_0.02pct' AND antibody == 'H3K27me3') HG19_ENCODE_BROAD;

Promoters  = SELECT(annotation_type == 'promoter') HG19_BED_ANNOTATION;

ActiveUnion = UNION() K4me1EtOH K27acEtOH;
ActiveOverlapping = COVER(2, ANY) ActiveUnion;
Active = MAP() ActiveUnion ActiveOverlapping;

RepressedUnion = UNION() K4me1EtOH K27me3EtOH;
RepressedOverlapping =  COVER(2, ANY) RepressedUnion;
Repressed = MAP() RepressedUnion RepressedOverlapping;

PoisedUnion = UNION() Repressed Active;
PoisedOverlapping = COVER(2, ANY) PoisedUnion;
Poised = MAP() PoisedUnion PoisedOverlapping;

ActiveInPromotersUnion = UNION() Active Promoters;
ActiveInPromotersOverlapping = COVER(2, ANY) ActiveInPromotersUnion;
ActiveInPromoters = MAP() ActiveInPromotersUnion ActiveInPromotersOverlapping;

ClosestK4me1FartherThan10k = JOIN(DGE(10000), MD(1); output: cat) ActiveInPromoters K4me1EtOH;

MATERIALIZE ClosestK4me1FartherThan10k INTO Result;
\end{lstlisting}

This job runned in about 2 minutes (113 sec), I obtained three sample, without duplicate.
The join is expliciting asking for one element, father than 10000 from the anchor in ``ActiveInPromoter'' from the ``K4me1EtOH''.

\end{document}