% !TEX TS-program = pdflatex
% !TEX encoding = UTF-8 Unicode

% This is a simple template for a LaTeX document using the "article" class.
% See "book", "report", "letter" for other types of document.

\documentclass[11pt]{article} % use larger type; default would be 10pt

\usepackage[utf8]{inputenc} % set input encoding (not needed with XeLaTeX)

%%% Examples of Article customizations
% These packages are optional, depending whether you want the features they provide.
% See the LaTeX Companion or other references for full information.

%%% PAGE DIMENSIONS
\usepackage{geometry} % to change the page dimensions
\geometry{a4paper} % or letterpaper (US) or a5paper or....
% \geometry{margin=2in} % for example, change the margins to 2 inches all round
% \geometry{landscape} % set up the page for landscape
%   read geometry.pdf for detailed page layout information

\usepackage{graphicx} % support the \includegraphics command and options

% \usepackage[parfill]{parskip} % Activate to begin paragraphs with an empty line rather than an indent

%%% PACKAGES
\usepackage{booktabs} % for much better looking tables
\usepackage{array} % for better arrays (eg matrices) in maths
\usepackage{paralist} % very flexible & customisable lists (eg. enumerate/itemize, etc.)
\usepackage{verbatim} % adds environment for commenting out blocks of text & for better verbatim
\usepackage{subfig} % make it possible to include more than one captioned figure/table in a single float
% These packages are all incorporated in the memoir class to one degree or another...

%%% HEADERS & FOOTERS
\usepackage{fancyhdr} % This should be set AFTER setting up the page geometry
\pagestyle{fancy} % options: empty , plain , fancy
\renewcommand{\headrulewidth}{0pt} % customise the layout...
\lhead{}\chead{}\rhead{}
\lfoot{}\cfoot{\thepage}\rfoot{}

%%% SECTION TITLE APPEARANCE
\usepackage{sectsty}
\allsectionsfont{\sffamily\mdseries\upshape} % (See the fntguide.pdf for font help)
% (This matches ConTeXt defaults)

%%% ToC (table of contents) APPEARANCE
\usepackage[nottoc,notlof,notlot]{tocbibind} % Put the bibliography in the ToC
\usepackage[titles,subfigure]{tocloft} % Alter the style of the Table of Contents
\renewcommand{\cftsecfont}{\rmfamily\mdseries\upshape}
\renewcommand{\cftsecpagefont}{\rmfamily\mdseries\upshape} % No bold!

\usepackage{listings}

%%% END Article customizations

%%% The "real" document content comes below...

\title{Genomic Computing}
\author{Simone Mosciatti}
%\date{} % Activate to display a given date or no date (if empty),
         % otherwise the current date is printed 

\begin{document}
\maketitle

\section{Exercise 1}

\subsection{A}

In the region there are two genes, I can see the beginning of SUPT3H in the negative strand and the beginning of RUNX2 on the positive strand.

\subsection{B}

It is possible to notice a peak of the GC Percent just at the very beginning of SUPT3H exactly in a CpG island.

Similarly we can see a peak of the GC Percent at the beginning of the third intron of RUNX2, which is, as well, inside a CpG island.

It is curius to notice that at the very beginning of RUNX2, roughly 100kb upstream the peak of the GC is way mildier.

The first CpG island following the positive strand, the one associated with SUPT3H is in position: chr6:45345186-45346261, with chromosome band: 6p21.1 and genomic size of 1076.

\section{Exercise 2}

\subsection{A}

Definitely no, the first three reads maps in two different regions, the fourth and the fifth maps in three different region while the last map in as much as four different regions.

\subsection{B / C}

There is an almost perfect match of all the reads with the gene 17 in this coordinates: chr17:41,256,928-41,258,548

The coordinates correspond to the fourth intron of gene BRCA1.

\subsection{C}

This reads, most likely, come from gene 17.

\end{document}